Desktop machines hebben een beperkte opslagcapaciteit en hebben weinig tot geen back-up mogelijkheden. Door de data te centraliseren op een fileserver kunnen we de capaciteit van de desktop uitbreiden zonder daar fysiek toegang tot te hebben. Daarnaast kunnen we de opgeslagen data op de fileserver eenvoudig centraal backuppen.

Een fileserver is een computer, server, in het netwerk die via een bepaald protocol bestanden aanbiedt aan gebruikers. In Microsoft Windows netwerken kan dit een SMB, Server Message Block, zijn en in een Linux omgeving bijvoorbeeld een NFS, Network File System, server zijn.

Het kenmerk van een fileserver is dat er een bepaald protocol (SMB, NFS) nodig is dat een bestandssysteem aanbiedt. Een gebruiker mount of mapped het filesysteem naar zijn lokale systeem en kan erop werken alsof het onderdeel is van zijn eigen machine.
