Direct Attached Storage is een 19"-behuizing met disken. Deze kast heeft meestal een SCSI of SAS aansluiting. Via software wordt het systeem als een grote disk of als verschillende partities aangeboden aan een host. De controller die de disken aanstuurt is meestal een RAID controller zodat data redundant wordt opgeslagen. De disken die de host ziet hebben geen enkel verband met de fysieke disken in de 19"-behuizing.

DAS en JBOD worden vaak door elkaar gebruikt. Sommigen maken het onderscheid door JBOD de oplossing zonder RAID te noemen en DAS een oplossing met RAID. Letterlijk vertaald: Direct verbonden opslag, maakt dat elke disk die direct aan een systeem verbonden is, intern of extern, onder de titel van DAS kan vallen. Vandaar vermoedelijk de verwarring over deze termen.
