Als je DAS of Direct Attached Storage googled kom je vaak uit bij een 19\inch-behuizing met disken. Deze kast heeft meestal een SCSI of SAS aansluiting. Via software wordt het systeem als een grote disk of als verschillende partities aangeboden aan een host. De controller die de disken aanstuurt is meestal een RAID controller zodat data redundant wordt opgeslagen. De "disken" die aan de host worden aangeboden hebben meestal geen enkel verband met de fysieke disken in de 19"-behuizing.

Ook als JBOD of Just a Bunch Of Disks googled kom je uit bij 19\inch-behuizingen met disken.

DAS en JBOD worden dan ook vaak door elkaar gebruikt. Sommigen maken het onderscheid door JBOD een oplossing zonder RAID te noemen en DAS een oplossing met RAID. In dit boek zullen we ons er verder niet zo druk om maken. Het is wel van belang dat je beide termen kent als aanduiding voor een systeem met veel disken.
