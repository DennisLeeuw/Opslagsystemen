Onder Storage verstaan we het aanbieden van opslagcapaciteit op opslagsystemen. Dit kan op harddisks in het systeem zijn op een fileserver die gekoppeld is aan het netwerk.

Door een bestandssysteem over het netwerk aan te bieden kan elk systeem dat het netwerk-protocol spreekt gebruik maken van de data, waarmee het delen van bestanden tussen verschillende systemen overbrugt wordt.

Met de opkomst van het Internet en speciaal het web (HTTP) zijn we data (documenten, plaatjes en filmpjes) met elkaar gaan delen, bijna elk apparaat spreekt tegen woorden HTTP of HTTPs en HTML (de opmaak taal van webdocumenten) en daarmee is een belangrijkdeel van de documenten ontsloten voor de hele wereld.

Dit document beschrijft de verschillende opslagsystemen en de technieken om deze systemen te koppelen en toegankelijk te maken.
