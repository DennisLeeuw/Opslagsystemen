NFS\index{NFS} stamt uit 1984 en is bedacht door Sun Microsystems. NFS is een netwerk bestandssysteem dat het mogelijk maakt om bestanden over het netwerk op een andere server te benaderen alsof het lokale bestanden zijn. Dit in tegenstelling met het tot dan toe veel gebruikte FTP\index{FTP} waarbij je de een server actief moet benaderen om een bestand te downloaden voordat je het gebruiken kan.

NFS heeft twee protocollen, een om het bestandssysteem te koppelen (mount) en een om het gemounte bestandssysteem te benaderen (nfs). NFS gebruikt een aantal Remote Procedure Calls (RPCs)\index{RPC} voor de toegang tot het bestandssysteem zoals het lezen en schrijven van bestanden.

NFS server houdt geen status bij van de clients, alleen welke client welke share gemount heeft. De client moet dus bij elke opdracht alle gegevens meesturen. Dit heeft als groot voordeel dat je een NFS server kan herstarten zonder dat clients hiervan een verstoring ondervinden. Een nadeel is dat NFS dus niet zelf bijhoud wie welk bestand benaderd en er dus twee gebruikers hetzelfde bestand kunnen schrijven. Het zogenaamde file-locking moet door een extern proces gebeuren (Network Lock Manager). Vanaf NFS 4 is locking wel een onderdeel van het protocol.
