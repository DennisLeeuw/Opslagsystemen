iSCSI is een methode om de SCSI commando's over een IP netwerk te versturen. De i staat dan ook voor Internet. Ethernet switches worden gebruikt om het netwerk op te zetten waarover TCP/IP en iSCSI gebruikt wordt om disken of partities te delen. Het is uitgevonden door Cisco en IBM en als standaard ingediend in 2000.

Het is mogelijk om iSCSI ook over het Internet te gebruiken, maar meestal wordt het gebruikt over ge\"isoleerde netwerken om de optimale perfomance uit het netwerk te halen en om de veilgheid van het SAN te waarborgen omdat standaard iSCSI geen encryptie gebruikt over het netwerk.

Clients (initiators) praten met storage devices (targets) door over een netwerk, meestal ethernet, SCSI commando's te sturen. Er zijn geen "disken" in SCSI, maar LUNs: Logical Unit Numbers. Een SCSI initiator praat met een target en adresseerd daarbij een LUN, wat een disk of een partitie zijn kan, maar ook een RAID5 oplossing. Voor SCSI is het alleen maar van belang dat het praat met een LUN en het is niet van belang wat er onder de motorkap gebeurd. Daarmee is een LUN een virtueel opslag systeem.

iSCSI kan een software matige oplossing zijn, wat betekent dat de initiator en/of de target volledig als software op uw systeem aanwezig is of er kan sprake zijn van een HBA (host bus adapter) die speciaal voor iSCSI gemaakt is. Er is ook nog een tussen oplossing dat is een iSCSI offload engine, die dan het iSCSI protocol voor zijn rekening neemt en ervoor zorgt dat het OS het minder druk krijgt. De HBA is vaak de duurste, maar ook de snelste oplossing en een complete software oplossing, zeker als dat met opens source software is, is de goedkoopste oplossing.
